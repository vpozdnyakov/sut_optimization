\documentclass{beamer}

\usetheme{HSE}
\setbeamertemplate{footline}[page number]
\usefonttheme{professionalfonts}

\usepackage{cmap}					
\usepackage{mathtext} 				
\usepackage[T2A]{fontenc}			
\usepackage[utf8]{inputenc}			
\usepackage[english,russian]{babel}
\usepackage{array}

\title{Квадратичная оптимизация в задаче прогнозирования SUT таблиц}
\author{Поздняков Виталий,  Рог Алексей}
\date{Декабрь 2019}

\begin{document}

\begin{frame}
\titlepage
\end{frame}

\section{Что такое SUT таблицы}

\begin{frame}
	\frametitle{\insertsection}

	\begin{center}
		\begin{tabular}{|c|c|c|c|c|}
			\hline
			& Отрасль 1 & Отрасль 2 & \dots & Отрасль N \\
			\hline
			Продукция 1	& \dots	& \dots	& \dots	& \dots \\
			Продукция 2	& \dots	& \dots	& \dots	& \dots \\
			\dots & \dots & \dots & \dots & \dots \\
			Продукция N	& \dots	& \dots	& \dots	& \dots \\
			\hline
		\end{tabular}
	\end{center}
	
	\begin{itemize}
    \item SUT (Supply and Use tables) — таблицы "затраты-выпуск"
    \item Supply Table — таблица \textbf{ресурсов} показывает сколько продукции было \textbf{реализовано} каждой отраслью в стране за год
    \item Use table — таблица \textbf{использования} показывает сколько продукции было \textbf{потреблено} каждой отраслью в стране за год
    \end{itemize}
    
\end{frame}

\section{Пример SUT таблиц}
\begin{frame}
	\frametitle{\insertsection}
	
	\begin{center}
		\footnotesize{
		\begin{tabular}{|l|l|l|l|}
			\hline
			& Сбор и 		& Строительство & Оптовая  	\\
			& переработка 	&  				& торговля  \\
			& отходов 		&  				&   		\\
			\hline
			\multicolumn{4}{|c|}{Таблица ресурсов} \\
			\hline
			Неметаллические минералы  	& 84 & 19 & 95 \\
			Основные металлы  			& 388 & 0 & 80 \\
			Металлоконструкции  		& 0 & 950 & 122 \\
			Строительные работы  		& 2 & 84100 & 73 \\
			\hline
			\multicolumn{4}{|c|}{Таблица использования} \\
			\hline
			Неметаллические минералы  	& 6 & 4128 & 0 \\
			Основные металлы  			& 56 & 944 & 0 \\
			Металлоконструкции  		& 41 & 5033 & 16 \\
			Строительные работы  		& 41 & 22250 & 216 \\
			\hline
		\end{tabular}
		}
	\end{center}
	
	
	Нидерланды, 2010 год, фрагмент

\end{frame}

\section{Задача прогнозирования SUT таблиц}
\begin{frame}
	\frametitle{\insertsection}
	
	\begin{itemize}
	\item Полное исследование требует финансовых и временных затрат, поэтому обычно проводится раз в 5 лет
	\item Требуется спрогнозировать SUT по историческим и сводным данным
	\item Исторические данные — это SUT прошлых лет
	\item Сводные данные: 
	\begin{itemize}
	    \item Итоговой объем продукции для каждой отрасли
	    \item Итоговой объем продукции для каждого вида продукции
	\end{itemize}
	\end{itemize}

\end{frame}

\section{Модель}
\begin{frame}
	\frametitle{\insertsection}
	
	\begin{itemize}
	\item $A_{m\times n}$ — матрица за прошлый год
	\item $X^{true}_{m\times n}$ — искомая поставок за текущий год
	\item Сводные данные за текущий год:
	\begin{itemize}
	\item Вектор $u = X^{true} \mathbf 1$ содержит сумму для $1, 2, \dots, m$ строки, \\ где $\mathbf 1$ — вектор единиц
	\item Вектор $v = (X^{true})^T \mathbf 1$ содержит сумму для $1, 2, \dots, m$ колонки
	\end{itemize}
	\item Требуется построить такую матрицу $X$, чтобы расстояние $f(X)$ до матрицы $A$ было минимальным при ограничениях: 	$$\begin{cases}
	X\mathbf 1 = u \\
	X^T\mathbf 1 = v \\
	x_{ij} \geq 0 \\
	x_{ij} = 0 \forall i,j:a_{ij} = 0
	\end{cases}$$
	\end{itemize}

\end{frame}

\section{Пример}
\begin{frame}
	\frametitle{\insertsection}
	
	Пусть матрица $A$ имеет вид:
	
	\begin{center}
		\begin{tabular}{|m{2cm}|m{2cm}|m{2cm}||m{2cm}|}
			\hline
			84 & 19 & 95 & \textbf{198} \\
			388 & 0 & 80 & \textbf{468} \\
			0 & 950 & 122 & \textbf{1072} \\
			2 & 84100 & 73 & \textbf{84175} \\
			\hline
			\hline
			\textbf{474} & \textbf{85069} & \textbf{370} & \textbf{Сумма} \\
			\hline
		\end{tabular}
	\end{center}
	
	Требуется построить матрицу $X$ с ограничениями на сумму по строкам и столбцам:
	
	\begin{center}
		\begin{tabular}{|m{2cm}|m{2cm}|m{2cm}||m{2cm}|}
			\hline
			? & ? & ? & \textbf{185} \\
			? & ? & ? & \textbf{495} \\
			? & ? & ? & \textbf{968} \\
			? & ? & ? & \textbf{88008} \\
			\hline
			\hline
			\textbf{473} & \textbf{88748} & \textbf{435} & \textbf{Сумма} \\
			\hline
		\end{tabular}
	\end{center}
	
\end{frame}

\section{Общая постановка задачи}
\begin{frame}
	\frametitle{\insertsection}
	
	$$f(X) \to \min (\max)$$
	$$g_i(X)\begin{bmatrix} = \\ \geq \\ \leq \end{bmatrix} 0, i = 1, 2, \dots, m$$
	
    \begin{itemize}
	\item Метод замены переменных
	\item Метод штрафных функций
	\item Метод множителей Лагранжа
	\item Релаксационный метод
	\end{itemize}
	
\end{frame}

\section{Метод замены переменных}
\begin{frame}
	\frametitle{\insertsection}
	
    В целевой функции выполняем замену $$x_i = \phi(z)$$ таким образом, чтобы безусловный экстремум новой функции совпадал с условным экстремумом целевой функции.
    
    Например
    $$f(x) = x^2 \to \min$$
    $$x \geq 1$$
    Выполним замену $$x = 1 + z^2$$
    Получаем $$(1 + z^2)^2 \to \min$$
	
\end{frame}

\section{Метод штрафных функций}
\begin{frame}
	\frametitle{\insertsection}
	
    В целевую функцию добавляем штрафную функцию, которая очень быстро растет (убывает) при нарушении ограничений.
    
    Например
    $$f(x) = x^2 \to \min$$
    $$x \geq 1$$
    Добавим штрафную функцию $$h(x) = (\min\{x-1, 0\})^2$$
    Тогда задача условной оптимизации сводится к безусловной
    $$x^2 + M \cdot h(x) \to \min$$ где $M>0$ — выбранная константа.
	
\end{frame}

\section{Метод множителей Лагранжа}
\begin{frame}
	\frametitle{\insertsection}
	
	Рассмотрим оптимизационную задачу
	$$f(X) \to \min$$
	При ограничениях
	$$\begin{matrix}
	g_1(X) = 0\\
	\dots \\
	g_k(X) = 0 \\
	\end{matrix}$$
	Функция Лагранжа имеет вид
	$$\mathcal L(X, \lambda_1, \dots, \lambda_k) = f(X) + \lambda_1g_1(X) + \dots + \lambda_kg_k(X)$$
	Решаем систему уравнений
	$$\nabla \mathcal L(X, \lambda_1, \dots, \lambda_k) = 0$$
	
\end{frame}

\section{Релаксационный метод}
\begin{frame}
	\frametitle{\insertsection}
	
	\begin{center}
	    \includegraphics[width=10cm]{diagram.png}
	    Возможные и прогрессивные направления
	\end{center}

\end{frame}

\section{Реализованные методы для прогнозирования SUT таблиц}
\begin{frame}
	\frametitle{\insertsection}
	
	\begin{itemize}
	\item Метод проектирования градиента (релаксационный)
	\item Статья "Projection of Supply and Use tables: methods and their empirical assessment" (Temurshoev, Yamano, Webb, 2010):
	\begin{itemize}
    	\item Improved squared differences (ISD)
    	\item \textbf{Improved normalized squared differences} (INSD)
    	\item Improved weighted squared differences (IWSD)
    	\item RAS
	\end{itemize}
	\end{itemize}
	
\end{frame}

\section{Метод Improved normalized squared differences}
\begin{frame}
	\frametitle{\insertsection}
	Сделаем замену
	$$z_{ij} = \frac{x_{ij}}{a_{ij}}$$
	Решаем оптимизационную задачу
	$$f(Z) = \sum_i \sum_j |a_{ij}|(z_{ij}-1)^2 \to \min$$
	При ограничениях
	$$\begin{matrix}
	\sum_j a_{ij}z_{ij} = u_{i} \\
	\sum_i a_{ij}z_{ij} = v_{j} \\
	a_{ij}z_{ij} \geq 0 \\
	\end{matrix}$$
	
\end{frame}

\section{Метод Improved normalized squared differences}
\begin{frame}
	\frametitle{\insertsection}
	
	Запишем функцию Лагранжа ($M$ — штрафной коэффициент)
	
	$$\mathcal{L}(Z, \lambda, \tau) = \frac{1}{2} \sum_i \sum_j |a_{ij}|\left((z_{ij}-1)^2 + M(\min\{0, z_{ij}\})^2\right)$$
	$$+ \sum_i \lambda_i\left(u_i - \sum_j a_{ij}z_{ij}\right) + \sum_j \tau_j \left(v_j - \sum_i a_{ij}z_{ij} \right)$$
	
	Требуется решить систему
	
	$$\begin{cases}
	\frac{\partial \mathcal L}{\partial z_{ij}} = 0 \\
	\sum_j a_{ij}z_{ij} = u_{i} \\
	\sum_i a_{ij}z_{ij} = v_{j}
	\end{cases}$$
	
\end{frame}

\section{Метод Improved normalized squared differences}
\begin{frame}
	\frametitle{\insertsection}
	
	Решение
	
	$$z_{ij} = 
	\begin{cases}
	1 & \text{если } a_{ij} = 0\\
	1 + \frac{a_{ij}}{|a_{ij}|(\lambda_i + \tau_i)} & \text{если } 1 + \frac{a_{ij}}{|a_{ij}|(\lambda_i + \tau_i)} \geq 0 \\
	\frac{1}{1+M}(1 + \frac{a_{ij}}{|a_{ij}|(\lambda_i + \tau_i)}) & \text{в остальных случаях}
	\end{cases}$$
	$$\lambda_i = \frac{1}{\sum_j|a_{ij}|}\left(u_i-\sum_ja_{ij} +\sum_j(M a_{ij} \min \{0, z_{ij}\} - \tau_j|a_{ij}|) \right)$$
	$$\tau_j = \frac{1}{\sum_i |a_{ij}| }\left(v_i-\sum_i a_{ij} +\sum_i(M a_{ij} \min \{0, z_{ij}\} - \lambda_i|a_{ij}|) \right)$$
	
\end{frame}

\section{Метод Improved normalized squared differences}
\begin{frame}
	\frametitle{\insertsection}
	
	Применим итерационный алгоритм:
	\begin{enumerate}
	    \item Инициализируем $Z^{[0]} = (1)$, $\lambda^{[0]} = (0)$, $\tau^{[0]} = (0)$
	    \item Вычисляем новые $\lambda^{[i]}, \tau^{[i]}$
	    \item Вычисляем новую матрицу $Z^{[i]}$
	    \item Повторяем 2-3 до тех пор пока не выполнится условие 
	    $$\begin{cases}
	    \lambda^{[i]} - \lambda^{[i-1]} < \epsilon_\lambda \\ 
	    \tau^{[j]} - \tau^{[j-1]} < \epsilon_\tau
	    \end{cases}$$
	    \item Выполняем обратную замену $x_{ij} = z_{ij}a_{ij}$
	\end{enumerate}
	
\end{frame}

\section{Метод проектирования градиента (релаксационный)}
\begin{frame}
	\frametitle{\insertsection}
	
	Запишем матрицу $X_{m \times n}$ в виде вектора 
	$$x = (x_{11}, x_{12}, \dots, x_{mn})^T$$
	Выполним замену переменных
	$$z_{i} = \frac{x_{i}}{a_{i}} - 1$$
	Решаем задачу квадратичной оптимизации
	$$f(Z) = z^T \cdot A^* \cdot z \to \min$$
	$$A^{**}z=b$$
	
	
\end{frame}

\section{Метод проектирования градиента (релаксационный)}
\begin{frame}
	\frametitle{\insertsection}
	$$f(Z) = z^T \cdot A^* \cdot z \to \min$$
	$$A^{**}z=b$$
	На примере матрицы $A_{2 \times 2}$:
	
	$$A^* = \begin{bmatrix}
	|a_{11}| & 0 & 0 & 0 \\
	0 & |a_{12}| & 0 & 0 \\
	0 & 0 & |a_{21}| & 0 \\
	0 & 0 & 0 & |a_{22}| \\
	\end{bmatrix}; 
	A^{**} = \begin{bmatrix}
	a_{11} & a_{12} & 0 & 0 \\
	0 & 0 & a_{21} & a_{22} \\
	a_{11} & 0 & a_{21} & 0 \\
	0 & a_{21} & 0 & a_{22} \\
	\end{bmatrix}$$
	$$b = \begin{bmatrix} u_1 & u_2 & v_1 & v_2 \end{bmatrix}^T - A^{**}\begin{bmatrix} 1 & 1 & 1 & 1 \end{bmatrix}^T$$
	
\end{frame}

\section{Метод проектирования градиента (релаксационный)}
\begin{frame}
	\frametitle{\insertsection}
	
	Переобозначим для краткости $A^{**} := A$. 
	
	Решение находится итерационно c начальной точкой
	$$Az_0 = b \Rightarrow z_0 = A^+b $$
	$$z_{i+1} = z_i + t_iD_i$$ где $D_i$ — направление шага, $t_i$ - длина шага
	
	$$D = -P \nabla f(Z) = -P 
	\begin{pmatrix} 
	2 \cdot |a_{11}| \cdot z_{11} \\
	... \\
	2 \cdot |a_{mn}| \cdot z_{mn} \\
	\end{pmatrix}$$
	
	$$P = I - A^T(AA^T)^{-1}A = I - A^+A$$
	
\end{frame}

\section{Пример}
\begin{frame}
	\frametitle{\insertsection}
	
	Имеем исходную матрицу $A$:
	
	\begin{center}
		\begin{tabular}{|m{2cm}|m{2cm}|m{2cm}||m{2cm}|}
			\hline
			84 & 19 & 95 & \textbf{198} \\
			388 & 0 & 80 & \textbf{468} \\
			0 & 950 & 122 & \textbf{1072} \\
			2 & 84100 & 73 & \textbf{84175} \\
			\hline
			\hline
			\textbf{474} & \textbf{85069} & \textbf{370} & \textbf{Сумма} \\
			\hline
		\end{tabular}
	\end{center}
	
	Если установить $M = 100$, $\epsilon_\lambda = \epsilon_\tau = 10^{-8}$, то за $95$ итерации матрица $X$ примет вид:
	
	\begin{center}
		\begin{tabular}{|m{2cm}|m{2cm}|m{2cm}||m{2cm}|}
			\hline
			72 & 14 & 99 & \textbf{185} \\
			398 & 0 & 97 & \textbf{495} \\
			0 & 826 & 142 & \textbf{968} \\
			2 & 87908 & 98 & \textbf{88008} \\
			\hline
			\hline
			\textbf{473} & \textbf{88748} & \textbf{435} & \textbf{Сумма} \\
			\hline
		\end{tabular}
	\end{center}
	
\end{frame}

\section{Другие методы}
\begin{frame}
	\frametitle{\insertsection}
	
	\begin{enumerate}
	    \item Improved squared differences
	    $$f(Z) = \sum_i \sum_j a_{ij}^2(z_{ij}-1)^2$$
	    \item Improved weighted squared differences
	    $$f(Z) = \sum_i \sum_j |a_{ij}^3|(z_{ij}-1)^2$$
	    \item RAS
	    $$f(Z) = \sum_i \sum_j |a_{ij}| \left(z_{ij} \ln \left( \frac{z_{ij}}{e} \right) + 1 \right)$$
	\end{enumerate}
	
\end{frame}

\section{Данные для тестирования}
\begin{frame}
	\frametitle{\insertsection}
	
	\begin{center}
	
	\end{center}
	\begin{itemize}
	    
	    \item SUT за 2010, 2011, 2012 годы (Нидерланды) 
	    
	    6 матриц 64x64
	    
	    Источник: World Input-Output Database (http://www.wiod.org/)
	    
	    \begin{itemize}
	        \item Первый сценарий: $2010 \Rightarrow 2011$
	        \item Второй сценарий: $(2010 \Rightarrow 2011) \Rightarrow 2012$
	    \end{itemize}
	    
	    \item SUT за 2012, 2013 годы (Россия)
	    
	    4 матрицы 62x59
	    
	    Источник: Росстат
	    
	    \begin{itemize}
	        \item Третий сценарий: $2012 \Rightarrow 2013$
	    \end{itemize}
	\end{itemize}
    
\end{frame}

\begin{frame}

    \includegraphics[width=\textwidth]{2019-12-12_05-25-06.png}
    
\end{frame}

\section{Метрики}
\begin{frame}
	\frametitle{\insertsection}
	
	\begin{itemize}
	    \item Mean absolute percentage error
	    
	    $$MAPE = \frac{1}{nm}\sum_i\sum_j\frac{|x_{ij} - x_{ij}^{true}|}{|x_{ij}^{true}|} \cdot 100$$
	    
	    \item Weighted absolute percentage error
	    
	    $$WAPE = \sum_i\sum_j\left( \frac{|x_{ij}^{true}|}{\sum_k \sum_l x_{kl}^{true}}\right) \frac{|x_{ij} - x_{ij}^{true}|}{|x_{ij}^{true}|}\cdot100$$
	    
	    \item Standardized weighted absolute difference
	    
	    $$SWAD = \frac{\sum_i \sum_j |x_{ij}^{true}| \cdot |x_{ij} - x_{ij}^{true}|}{\sum_k \sum_l\left( x_{kl}^{true} \right)^2}$$
	    
	\end{itemize}
	
\end{frame}

\section{Результаты. Нидерланды, зависимость от M}
\begin{frame}
	\frametitle{\insertsection}
	
	\begin{center}
		$M=0$, Supply $2010 \Rightarrow 2011$
		\begin{tabular}{|r|m{2cm}|m{2cm}|m{2cm}|m{2cm}|}
			\hline
			& \textbf{IWSD} & \textbf{INSD} & \textbf{ISD} & \textbf{RAS} \\
			\hline
			\textbf{MAPE} & 852.87 & 7.58 & 191.20 & 81.94 \\
			\textbf{WAPE} & 16.03 & 1.70 & 10.94 & 5.83 \\
			\textbf{SWAD} & 0.05 & 0.00 & 0.05 & 0.05 \\
			\hline
		\end{tabular}
	\end{center}
	
	\begin{center}
		$M=10^5$, Supply $2010 \Rightarrow 2011$
		\begin{tabular}{|r|m{2cm}|m{2cm}|m{2cm}|m{2cm}|}
			\hline
			& \textbf{IWSD} & \textbf{INSD} & \textbf{ISD} & \textbf{RAS} \\
			\hline
			\textbf{MAPE} & 319.21 & 7.58 & 157.43 & 81.94 \\
			\textbf{WAPE} & 13.68 & 1.70 & 10.66 & 5.83 \\
			\textbf{SWAD} & 0.05 & 0.00 & 0.05 & 0.05 \\
			\hline
		\end{tabular}
	\end{center}
	
\end{frame}

\section{Результаты. Нидерланды, при $M = 10^5$}
\begin{frame}
	\frametitle{\insertsection}
	
	\includegraphics[width=\textwidth]{2019-12-12_06-04-50.png}
	
\end{frame}

\section{Результаты. Россия, при $M = 1000$}
\begin{frame}
	\frametitle{\insertsection}
	
	\begin{center}
		Supply $2012 \Rightarrow 2013$
		\begin{tabular}{|r|m{1.5cm}|m{1cm}|m{1.2cm}|m{1cm}|m{1cm}|}
			\hline
			& \textbf{IWSD} & \textbf{INSD} & \textbf{ISD} & \textbf{RAS} & \textbf{GRAD} \\
			\hline
			\textbf{MAPE} & 26898.99 & 183.51 & 7512.82 & 245.85 & 160.42 \\
			\textbf{WAPE} & 18.21 & 1.86 & 15.35 & 9.91 & 1.92 \\
			\textbf{SWAD} & 0.07 & 0.00 & 0.06 & 0.09 & 0.00 \\
			\textbf{Time (sec)} & 303.64 & 64.31 & 4.06 & 5.41 & 0.58 \\
			\hline
		\end{tabular}
	\end{center}
	
	\begin{center}
		Use $2012 \Rightarrow 2013$
		\begin{tabular}{|r|m{1.5cm}|m{1cm}|m{1.2cm}|m{1cm}|m{1cm}|}
			\hline
			& \textbf{IWSD} & \textbf{INSD} & \textbf{ISD} & \textbf{RAS} & \textbf{GRAD} \\
			\hline
			\textbf{MAPE} & 31852.40 & 11.27 & 1715.71 & 25.68 & 12.80 \\
			\textbf{WAPE} & 26.08 & 5.52 & 16.54 & 10.50 & 8.23 \\
			\textbf{SWAD} & 0.10 & 0.03 & 0.09 & 0.10 & 0.06 \\
			\textbf{Time (sec)} & 264.08 & 0.29 & 3.97 & 0.01 & 0.58 \\
			\hline
		\end{tabular}
	\end{center}
	
\end{frame}

\end{document}
